\documentclass[11pt]{article}
\usepackage{ctex}
\usepackage{amssymb}
\usepackage{amsmath}
\usepackage{hyperref}


\begin{document}
\title{随机过程第一次Project代码清单}\footnote{代码均在code文件夹}
%\subtitle{概率论与随机过程(2)  第一次大作业}
\author{无47\hspace{2em}刘前\thanks{清华大学电子工程系(E-mail: liuqian14@mails.tsinghua.edu.cn)}\hspace{2em}2014011216\hspace{2em}}
\maketitle

\section{Part A: Metropolis-Hastings算法应用}
\begin{table}[!htbp]
  \centering  
  \begin{tabular}{c|c} 
  \hline\hline
  文件名 & 文件功能\\ \hline
  bigauss.m& 给定高斯分布的概率密度函数 \\ \hline
  MH\_Origin.m & MH算法的初步实现 \\
  MH\_Origin\_Test.m & MH算法的初步实现(测试性能) \\ \hline
  MH\_Sigma.m & 对MH算法中提议函数方差的探究 \\
  MH\_Sigma\_Test.m &  比较提议函数方差不同时MH算法的性能\\ \hline
  MH\_SampleNum.m & 对MH算法中随机样本数目的探究 \\
  MH\_SampleNum\_Test.m &  比较随机样本数不同时MH算法的性能\\ \hline
  MH\_Step.m & 对MH算法中随机样本间隔选取的探究 \\
  MH\_Step\_Test.m &  比较选取随机样本间隔不同时MH算法的性能\\ \hline
  MH\_Gibbs.m &  实现Gibbs算法 \\ 
  MH\_Gibbs\_Test.m &  测试Gibbs算法的性能 \\\hline
  MH\_Optimal.m & 各参数优化后的MH算法 \\
  MH\_Compare.m & 对优化前后的算法进行对比\\ \hline\hline
\end{tabular}
\caption{Part A代码清单}
\label{CodeList1}
\end{table}


\section{Part B: RBM模型归一化常数的估计}

\begin{table}[!htbp]
  \centering  
  \begin{tabular}{c|c} 
  \hline\hline
  文件名 & 文件功能\\ \hline
  AIS2.m  & AIS算法实现 \\
  TAP2.m & TAP算法(二阶近似)实现\\
  TAP3.m & TAP算法(三阶近似)实现\\
  RTS.m & RTS算法实现 \\
  LikeliHood.m & 计算测试数据上的总似然值 \\
  Compare.m & 比较各算法的性能 \\
  run.m & 产生要求的z.mat \\
  \hline\hline
\end{tabular}
\caption{Part B代码清单--run}
\label{CodeList2-run}
\end{table}


\begin{table}[!htbp]
  \centering  
  \begin{tabular}{c|c} 
  \hline\hline
  数据文件名 & 数据\\ \hline
  h10.mat & 隐变量为10的RBM模型 \\ 
  h20.mat & 隐变量为20的RBM模型 \\
  h100.mat & 隐变量为100的RBM模型 \\
  h500.mat & 隐变量为500的RBM模型 \\ \hline
  test.mat & 测试数据 \\
  train.mat & 训练数据 \\
  \hline\hline
\end{tabular}
\caption{Part B代码清单--Data}
\label{CodeList2-Data}
\end{table}


\begin{table}[!htbp]
  \centering  
  \begin{tabular}{c|c} 
  \hline\hline
  文件名 & 文件功能\\ \hline
  AIS1.m  & AIS算法实现(无分批数据) \\
  AIS2.m  & AIS算法实现(有分批数据) \\ 
  AIS\_Test.m & AIS算法性能测试 \\ \hline
  LikeliHood.m & 计算测试数据上的总似然值 \\ \hline
  AIS\_Beta.m & 探究$\beta$参数对AIS算法的影响\\
  AIS\_Mrun.m & 探究执行次数$M$对AIS算法的影响\\
  AIS\_Data.m & 探究是否使用分批数据对AIS算法的影响\\ 
  \hline\hline
\end{tabular}
\caption{Part B代码清单--AIS}
\label{CodeList2-AIS}
\end{table}

\begin{table}[!htbp]
  \centering  
  \begin{tabular}{c|c} 
  \hline\hline
  文件名 & 文件功能\\ \hline
  TAP2.m & TAP算法(二阶近似)实现\\
  TAP3.m & TAP算法(三阶近似)实现\\
  TAP\_Test.m & TAP算法测试 \\
  LikeliHood.m & 计算测试数据上的总似然值 \\
  \hline\hline
\end{tabular}
\caption{Part B代码清单--TAP}
\label{CodeList2-TAP}
\end{table}

\begin{table}[!htbp]
  \centering  
  \begin{tabular}{c|c} 
  \hline\hline
  文件名 & 文件功能\\ \hline
  RTS.m & RTS算法实现\\
  RTS\_Test.m & RTS算法测试 \\
  \hline\hline
\end{tabular}
\caption{Part B代码清单--RTS}
\label{CodeList2-RTS}
\end{table}


\begin{table}[!htbp]
  \centering  
  \begin{tabular}{c|c} 
  \hline\hline
  文件名 & 文件功能\\ \hline
  sigm.m & Sigmoid函数 \\
  sigmrnd.m & 生成服从Sigmoid函数的随机数 \\
  rbmtrain.m & RBM模型的训练函数\\
  dbnsetup.m & DBN\footnote{DBN, Deep Belief Networks,深度信念网络}的建立\\
  dbntrain.m & DBN的训练\\
  RBM\_Training.m & 执行RBM模型训练,得到h*.mat\\ \hline
  AIS2.m & AIS算法估计归一化常数 \\ 
  LikeliHood.m & 计算测试数据上的总似然值 \\
  trained\_run.m & 估计归一化常数并计算似然值 \\
  \hline\hline
\end{tabular}
\caption{Part B代码清单--train}
\label{CodeList2-train}
\end{table}


\end{document}